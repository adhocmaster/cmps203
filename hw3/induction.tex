\documentclass[a4paper, 11pt]{article}
\usepackage{amssymb}
\usepackage{bussproofs}
\usepackage{charter}
\usepackage[top=1in, bottom=1.25in, left=1in, right=1in]{geometry}

\title{HW3: Induction}
\author{Golam Md Muktadir}
\begin{document}
\maketitle

\begin{prooftree}
  \AxiomC{$\langle x,\sigma \rangle \Downarrow \sigma(x)$}
  \AxiomC{$\langle y,\sigma \rangle \Downarrow \sigma(y)$}
  \AxiomC{$\langle z,\sigma \rangle \Downarrow \sigma(z)$}
  \BinaryInfC{$\langle y + z,\sigma \rangle \Downarrow \sigma(y) + \sigma(z)$}
  \BinaryInfC{$\langle x * (y + z),\sigma \rangle \Downarrow \sigma(x) * (\sigma(y)+\sigma(z))$}
\end{prooftree}

\section{Problem 1}

\subsection{Question}
In the WHILE language, prove or disprove the equivalence of the two commands:
$$t := x; x := y; y := t$$
and
$$t := y; y := x; x := t$$
(where x; y, and t are distinct locations).
\subsection{Answer}
We can disprove it with a counter example with a state $s$ in which two sequences 
yield different results.

Let 
$$s = [t\rightarrow0, x\rightarrow5, y\rightarrow100]$$

Then,
$$ \langle t := x; x := y; y := t, s \rangle \Downarrow s[t\rightarrow5, x\rightarrow100, y\rightarrow5] $$
$$ \langle t := y; y := x; x := t, s \rangle \Downarrow s[t\rightarrow100, x\rightarrow100, y\rightarrow5] $$
So, location, $t$, has different values after execution of the sequences under this $s$.
\section{Problem 2}

\subsection{Question}
In the WHILE language, prove that if

\begin{prooftree}
  \AxiomC{$\langle \textrm{while b do } y := y-x, s \rangle \Downarrow s'$}
\end{prooftree}
then there exists an integer k such that
  $$s(y) = s'(y) + k * s(x)$$
Please make it explicit if/when you reason by induction on derivations, stating your induction hypoth-
esis.

\subsection{Answer}

We can prove it by induction on the derivation of while command.\newline 
Let us define, 

W $\doteq \textrm{while b do } y := y-x $

$\sigma_i \doteq s[y := y -(k-i)*x]$

P(n) $\doteq \langle W, \sigma_n \rangle \Downarrow \sigma_o$ \newline
So,

$\sigma_o = s[y := y -k*x] = s'$

$\sigma_k = s[y := y -x]$ 
\\[5mm]
\textbf{Base case: $i = 0 $}
\begin{prooftree}

  \AxiomC{$\langle b, \sigma_o \rangle \Downarrow false$}
  \AxiomC{$\langle skip, \sigma_o \rangle \Downarrow \sigma_o$}
    \LeftLabel{D:}
    \RightLabel{\scriptsize( by $if_{sos}^{ff}, while_{sos}$ )}
    \BinaryInfC{$\langle  W, \sigma_o \rangle \Downarrow \sigma_o$}
  
\end{prooftree} 
So, P(0) holds.
\\[5mm]
\textbf{Induction case: $\forall i <= n, P(i)$ holds, $n \in \mathbb{N}$}
We need to show that P(n+1) holds, too.

\begin{prooftree}
  
  \AxiomC{$D1: \langle b, \sigma_o \rangle \Downarrow true$}
  \AxiomC{$D2: \langle y := y - x, \sigma_{n+1} \rangle \Downarrow \sigma_n$}
  \AxiomC{$D': \langle W, \sigma_n \rangle \Downarrow \sigma_o$}
    \LeftLabel{D:}
    \RightLabel{\scriptsize( by $while_{sos}$ )}
    \TrinaryInfC{$\langle  W, \sigma_{n+1} \rangle \Downarrow \sigma_{o}$}
  
\end{prooftree}
Derivation of D2:
\begin{prooftree}
  
  \AxiomC{$\langle y := y - x, s[y := y -k*x + (n+1)*x] \rangle \rightarrow s[y := y -k*x + (n)*x]$}
    \RightLabel{\scriptsize( by $ass_{sos}$ )}
    \LeftLabel{D2:} 
    \UnaryInfC{$\langle y := y - x, \sigma_{n+1} \rangle \Downarrow \sigma_n$}
  
\end{prooftree}
Now D' is true by our induction hypothesis. So, P(n+1) holds. 
\\[5mm]
So, we proved $\sigma_o$ to be our terminal state. We can see from the definition of it that,

$\sigma_o(y) = s(y) - k*x $ \newline
$\Rightarrow s'(y) = s(y) - k*x $\newline
$\Rightarrow s(y) = s'(y) + k*x $ (Proved)

\section{Problem 3}

\subsection{Question}
In the WHILE language, prove:
$$\forall c1, c2, c3 : c1; (c2; c3) \approx (c1; c2); c3$$

\subsection{Answer}

$\langle  \rangle \Downarrow $
\begin{prooftree}
      \AxiomC{$\langle c1, s_o \rangle \Downarrow s'$}
        \AxiomC{$\langle skip, s' \rangle \Downarrow s'$}
          \AxiomC{$\langle c1, s' \rangle \Downarrow s''$}
          \AxiomC{$\langle c2, s'' \rangle \Downarrow s $}
        \BinaryInfC{$\langle c2, c3, s' \rangle \Downarrow s$}
      \RightLabel{\scriptsize(seq)}
      \BinaryInfC{$LS': \langle (c2, c3), s' \rangle \Downarrow s$}
    \LeftLabel{LS:} \RightLabel{\scriptsize(seq)}
    \BinaryInfC{$\langle c1; (c2; c3), s_o \rangle \Downarrow s$}
  
\end{prooftree}
So, replacing LS' with commands changing states only in its derivation sequence, we get

\begin{prooftree}
  \AxiomC{$ L1: \langle c1, s_o \rangle \Downarrow s'$}
  \AxiomC{$ L2: \langle c1, s' \rangle \Downarrow s''$}
  \AxiomC{$ L3: \langle c2, s'' \rangle \Downarrow s $}
\LeftLabel{LS:} \RightLabel{\scriptsize(seq)}
\TrinaryInfC{$\langle c1; (c2; c3), s_o \rangle \Downarrow s$}

\end{prooftree}

\begin{prooftree}
        \AxiomC{$\langle skip, s_o \rangle \Downarrow s_o$}
          \AxiomC{$\langle c1, s_o \rangle \Downarrow t'$}
          \AxiomC{$\langle c2, t' \rangle \Downarrow t''$}
        \BinaryInfC{$\langle c2, c3, s_o \rangle \Downarrow t'$}
      \RightLabel{\scriptsize(seq)}
      \BinaryInfC{$RS': \langle (c1, c2), s_o \rangle \Downarrow t''$}
      \AxiomC{$\langle c3, t'' \rangle \Downarrow t$}
    \LeftLabel{RS:}
    \BinaryInfC{$\langle (c1; c2); c3, s_o \rangle \Downarrow t$}
  
\end{prooftree}
So, replacing RS' with commands changing states only in its derivation sequence, we get

\begin{prooftree}
    \AxiomC{$ R1: \langle c1, s_o \rangle \Downarrow t'$}
    \AxiomC{$ R2: \langle c2, t' \rangle \Downarrow t''$}
    \AxiomC{$ R3: \langle c3, t'' \rangle \Downarrow t$}
  \LeftLabel{RS:}
  \TrinaryInfC{$\langle (c1; c2); c3, s_o \rangle \Downarrow t$}

\end{prooftree}
Now, because of determinism of L1 and R1, we have, \newline
$ s' = t' $\newline
$\Rightarrow R2 = \langle c2, s' \rangle \Downarrow t''$\newline
$\Rightarrow s'' = t'' $ {\scriptsize(determinisim of L2 and R2)} \newline
$\Rightarrow R3 = \langle c3, s'' \rangle \Downarrow t$\newline
$\Rightarrow s = t $ {\scriptsize(determinisim of L3 and R3)} \newline
Thus, $\langle c1; (c2; c3), s_o \rangle \Downarrow s \approx \langle (c1; c2); c3, s_o \rangle \Downarrow t$\newline
$\Rightarrow c1; (c2; c3) \approx (c1; c2); c3$ (Proved)
\end{document}
